\documentclass[a4]{article}
\usepackage{geometry}
\geometry{a4paper,left=2cm,right=3cm, top=2cm, bottom=2cm} 
%\usepackage[austrian]{babel}
\renewcommand{\familydefault}{\sfdefault}
\usepackage{amsfonts,latexsym,amssymb,graphicx}
\usepackage{subfigure,epsfig,epstopdf}
%\usepackage{pdfsync}
\usepackage[utf8]{inputenc}
%\usepackage[T1]{fontenc}
\usepackage{booktabs} % For professional looking tables
\usepackage{multirow}
\usepackage{amsmath}

\usepackage[section]{placeins}

\title{\bf 183.605 \\ Machine Learning for Visual Computing \\ Assignment 2}
\author{Group 12: \\
	Hanna Huber (0925230) \\ Lena Trautmann (1526567) \\ Elisabeth Wetzer (0726681)}
\date{\today}


\begin{document}
\maketitle
\noindent

\section{Assignment 2}
\subsection{The dual optimization problem}

\noindent {\bf Tasks:}
\begin{itemize}
\item Generate a suitable training set of linearly separable data
\end{itemize}
The training set is generated in \texttt{generateTrainingData.m}. The function \texttt{generateTrainingData(N, xRange, yRange, linear)} takes the number of sample points, the domains (defined by a lower and an upper bound) from which the $x$ and $y$ coordinates are sampled and a flag which indicates if the resulting data should be linearly separable. A set of random 2D coordinates is created using the MATLAB function \texttt{rand}. Linear separability is achieved by labelling the according to the condition $ x_i + y_i > \bar{x} + \bar{y}$, where $\bar{x}$ and $\bar{y}$ denote the domain centers.

\begin{itemize}
\item Plot the input vectors in $\mathbb{R}^2$ and visualize corresponding target values (e.g. by using color). 
\end{itemize}
The resulting training data is displayed in Figure~\ref{fig:linearData}. Sample points with class label 1 are marked red, points with label -1 are merked green.
\begin{figure}[!h]
	\begin{center}
		\centering
		\includegraphics[width=6cm]{../figures/linearData.pdf}
	\end{center}	
	\caption{Linearly separable data with color-coded class labels (1=red, -1=green) for $N=100$.}
	\label{fig:linearData}
\end{figure}

\begin{itemize}
\item Visualize the support vectors and plot the decision boundary.
\end{itemize}
Figure~\ref{fig:sv} shows the support vectors defined by \texttt{trainSVM}.
\begin{figure}[!h]
	\begin{center}
		\centering
		\includegraphics[width=6cm]{../figures/sv.pdf}
	\end{center}
	\caption{Training data with support vectors marked by blue circles and decision boundary plotted in blue.}
	\label{fig:sv}
\end{figure}


\subsection{The kernel trick}

\noindent {\bf Tasks:}
\begin{itemize}
\item Try different values for $\sigma$ (the RBF parameter).
\end{itemize}
\begin{figure}[h!]
\centering
	\subfigure[$\sigma=0.05$]{\includegraphics[width=5cm]{../figures/sv_kernel5.pdf}}
	\subfigure[$\sigma=0.5$]{\includegraphics[width=5cm]{../figures/sv_kernel50.pdf}}
	\subfigure[$\sigma=1$]{\includegraphics[width=5cm]{../figures/sv_kernel100.pdf}}
	\\
	\subfigure[$\sigma=4$]{\includegraphics[width=5cm]{../figures/sv_kernel400.pdf}}
	\subfigure[$\sigma=7.06$]{\includegraphics[width=5cm]{../figures/sv_kernel706.pdf}}
	\subfigure[$\sigma=7.07$]{\includegraphics[width=5cm]{../figures/sv_kernel707.pdf}}
\caption{Training data with support vectors marked by blue circles and decision boundary plotted in blue, for different values of the RBF parameter $\sigma$.}
\label{fig:sv_kernel}
\end{figure}
The radial basis function kernel is defined by $K(x,y) = exp(-\frac{\|x-y\|^2}{\sigma^2})$.
The corresponding support vectors are shown in Figure~\ref{fig:sv_kernel} for different values of the RBF parameter $\sigma$. We can see, that the decision boundary gets better for larger $\sigma$ up to a value of 7.07. With $\sigma>=7.07$ the values of $\alpha$ are all greater than $10^{-8}$, hence we have all data points as support vectors. Furthermore, the least number of SVs was determined for $\sigma=7.06$.
%In the lecture notes about RBF-networks, we discussed a selection of $\sigma= 2*avgdist$, where $avgdist$ denotes the average distance of the centers. Having distances between the data points of approximately 10 units (or slightly more), $\sigma= 25$ is then selected.

\begin{itemize}
\item Generate a non-linearly separable training set, plot the data, visualize the support vectors and plot the decision boundary.
\end{itemize}
See also Figure~\ref{fig:sv_kernel}


%\begin{figure}[h!]
%\centering
%	\subfigure[]{\includegraphics[width=5cm]{../figures/originalBatchIt1.pdf}}
%	\subfigure[]{\includegraphics[width=5cm]{../figures/originalBatchIt342.pdf}}
%	\subfigure[]{\includegraphics[width=5cm]{../figures/originalBatchIt685.pdf}}
%\caption{Perceptron decision boundary in the original data space at iterations \#1, \#342 and \#685 of batch learning.}
%\label{fig:origBA}
%\end{figure}


% \bibliography{lit}
% \bibliographystyle{unsrt}

\end{document}
